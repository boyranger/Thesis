%\chapter{INDTRODUCTION}
%This chapter includes the following subtopics, namely: 
%(1) Rationale; 
%(2) Theoretical Framework;
% (3) Conceptual Framework/Paradigm; 
%(4) Statement of the problem; 
%(5) Hypothesis (Optional); 
%(6) Assumption (Optional); 
%(7) Scope and Delimitation;  and 
%(8) Importance of the study.

% =========================================================
%\section{Rationale}
% =========================================================
%This section have to explain: (a) the background of the study; (b) describe the problem situation considering global, national and local forces; (c) Justify the existence of the problem situation by citing statistical data and authoritative sources; and (d) Make a clinching statement that will relate the background to the proposed research problem.

% =========================================================
%\section{Theoretical Framework}
% =========================================================
%Discuss the theories and/or concepts, which are useful in conceptualizing the research.

% =========================================================
%\section{Conceptual Framework/Paradigm}
% =========================================================
%Identify and discuss the variables related to the problem, and present a schematic diagram of the paradigm of the research and discuss the relationship of the elements/variables therein.

% =========================================================
%\section{Statement of the Problem}
% =========================================================
%The general problem must be reflective of the title. It should be stated in such a way that it is not answerable by yes or no, not indicative of when and where. Rather, it should reflect between and among variables. Each sub-problem should cover mutually exclusive dimensions (no overlapping). The sub-problem should be arranged in logical order from actual to analytical following the flow in the research paradigm.

% =========================================================
%\section{Objective and Hypotheses}
% =========================================================
%First, explain the objective according the problem, then hypothesis. A hypothesis should be measurable/ desirable. It expresses expected relationship between two or more variables. It is based on the theory and/or empirical evidence. There are techniques available to measure or describe the variables. It is on a one to one correspondence with the specific problems of the study. A hypothesis in statistical form has the following characteristics: (a) it is used when the test of significance of relationships and difference of measures are involved; and (b) the level of significance if stated.

% =========================================================
%\section{Assumption}
% =========================================================
%An assumption should be based on the general and specific problems. It is stated in simple, brief, generally accepted statement.

% =========================================================
%\section{Scope and Delimitation}
% =========================================================
%Indicate the principal variables, locale, timeframe, and justification.

% =========================================================
%\section{Significance of the Study}
% =========================================================
%This section describes the contributions of the study as new knowledge, make findings more conclusive. It cites the usefulness of the study to the specific groups.


% =========================================================
%\section{Latar belakang masalah}
% =========================================================
%Berisi landasan permasalahan yang diperkuat dengan sitasi dari literature (Paper conference/paper jurnal/textbook 3 tahun terakhir). Permasalahan dapat diambil dari penelitian/pekerjaan sebelumnya dalam 3 tahun terakhir. Berisi minimal 1.5 halaman. Pada alinea terakhir, nyatakan kaitan antara Tesis yang akan dibuat dengan penelitian/pekerjaan sebelumnya.
 
 %Back ground should contain reasons of why phenomenon is chosen as the research topic; 
 %
 \begin{enumerate}
 	\item Write down your research problem
 	
 	bagaimana penerapan behavior based kontrol menggunakan neuro-fuzzy Q-learning pada \textit{Autonomous Trash Collector Robot} (ATCR)  untuk  navigasi  robot  pada lingkungan dinamik.
 	
 	\item Identify major topics to cover in your research study (the phenomenon, previous/recent technology, the trend of technology)
 	
 	Setiap tahunnya, lebih dari 2 (dua) juta ton plastik dibuang ke sungai dan akhirnya hanyut ke laut \cite{Othman2020}, sehingga sistem pembuangan sampah menjadi sektor yang cukup krusial \cite{Othman2020}\cite{Hossain2019}. Metode pengumpulan sampah secara manual menjadi metode yang sering digunakan untuk mengatasi krisis tersebut\cite{Khan2020}. Namun, terdapat beberapa masalah pada pengelolaan sampah secara manual, keselamatan para tenaga kerja, tidak dapat menjaungkau daerah terpencil, biaya pengorprasian, dan lainya\cite{Khan2020}. Autonomous Robot menjadi solusi untuk mengatasi permasalahan tersebut, karena dapat mengurangi resiko kecelakaan, dapat menjangkau daerah terpencil, dan dapat melakukan pekerjaan secara berulang \cite{Khan2020}\cite{Bai2018}. Autonomous Robot telah banyak dikembangkan pada beberape penelitian, seperti: Robot pembersih dinding \cite{HouxiangZhang2006}, pembersih air\cite{Yuan2011}, dan pembersih lantai\cite{Bai2018}\cite{Kang2014}\cite{Palacin2004}. Berdasarkan pada sistem kerja Autonomous Robot, sistem pengelola sampah dapat dibuat secara otomatis\cite{Bai2018}\cite{Nagayo2019}\cite{Prasetyo2020}.
 	
 	%the gap of existing condition and the future condition;
 	%who,where,when (auth,paper,year)
 	%what (problem)
 	%how(method)
 	%
 	Beberapa penelitian terkait Autonomous Robot untuk pengolahan sampah telah banyak dilakukan pada beberapa penelitian. Salah satunya pada penelitian yang dilakukan oleh Aditya P. P. Prasety, etc \cite{Prasetyo2020}, di mana pada penelitian tersebut Autonomous Robot dibuat menggunakan lengan manipulator untuk mengambil sampah. Posisi robot dikendalikan berdasarkan sensor ultrasonic dan sistem navigasi. Robot tersebut mampu membersihkan dua jenis sampah sebanyak 40 kali. Penelitian lain serupa juga dilakukan oleh Muhammad Abbas Khan, et al \cite{Khan2020}, dimana pada penelitian tersebut Robot Semi-autonomous dibuat untuk mengambil sampah berdasarkan perintah dari perangkat Smartphone melalui bluetooth. Robot juga dilengkapi sensor ultrasonic sehingga mampu mendeteksi posisi sampah. Pada Referensi \cite{Bai2018}, sebuah Autonomous Robot dibuat untuk mengambil sampah yang beroperasi di rumput. Robot mampu mendeteksi sampah secara otomatis dan akurat dengan menggunakan algoritma Deep Neural Network \cite{Kong2009}. Robot tersebut dilengkapi sensor ultrasonic\cite{Michael2008} dan sistem navigasi \cite{Wang2008}. Hasil pengujian menunjukan bahwa robot mampu mendeteksi sampah dengan akurasi 95\%. Pada makalah\cite{Arai2019}, Autonomous Robot dirancang menggunakan algoritma Convolutional Neural Network (CNN) untuk mendeteksi berbagai jenis sampah, seperti kaleng, botol plastik, dan kotak makan siang secara otomatis. Sistem robot mampu mendeteksi sampah secara outdoor. Berdasarkan data hasil pengujian\cite{Arai2019}, robot tersebut mampu mendeteksi sampah dengan tingkat kepresisian sebesar 95,6\% dan tingkat akurasi sebesar 96,8\% .
 	
 	
 	%the problem identification; metode yang d usulkan dan kelemahan yg ada
 	robot dapat bergerak secara mandiri berdasarkan lingkungannya seperti [5] dan bergerak manual berdasarkan perintah dari operator seperti [6]. Salah satu cara untuk mendapatkan informasi di sekitarnya adalah dengan menggunakan teknik pengindraan visual yang tidak membutuhkan banyak sensor [4], [8].
 	Beberapa teknik navigasi robot AGV telah diaplikasikan, yaitu navigasi berbasis perilaku (behaviors), petunjuk daerah (landmark), dan berbasis
 	penglihatan (vision) [4]. Navigasi berbasis perilaku lebih cocok digunakan di lingkungan yang tidak terstruktur. Navigasi berbasis landmark membutuhkan
 	sebuah tanda, misalnya garis. Navigasi berbasis penglihatan menggunakan kamera untuk mengetahui lingkungan di sekitarnya, misalnya warna dan jarak
 	
 	%edit sesuaikan ATCS
 	
 	
 	Terispirasi dari makalah [4] dan makalah [14], thesis ini bertujuan untuk merancang \textit{Autonomous Trash Collector System} (ATCS) menggunakan \textit{Deep Reinfocement Learning}. Robot dilengkapi dengan sensor lidar dan sistem navigasi untuk mendeteksi posisi robot. Selain itu, ATCS juga dilengkapi sistem kendali \textit{Adaptive Neuro-Fuzzy Inference System} (ANFIS) sebagai sistem kendali pada motor\cite{Saputra2019}. Penelitian ini diharapkan dapat membantu mengatasi masalah pengelolaan sampah yang kebanyakan masih dikelola secara manual.
 	
 	\item Write a research objective for the topic your study is researching.
 	
 	penerapan behavior based kontrol menggunakan neuro-fuzzy q-learning pada real sistem untuk navigasi dan kontrol autonomous mobile robot.
 	
 \end{enumerate}


 
 

%NEURO-FUZZY DEEP Q-LEARNING


%possibility that the phenomenon will give new concepts as a result; 

% =========================================================
%\section{Rumusan masalah}
% =========================================================

%Berisi urutan permasalahan yang dihadapi untuk menyelesaikan penelitian, dalam 1 kalimat. Buat dalam bentuk point rumusan masalah dan berisi minimal ½  halaman.
%Problem identification, Objective, Relation between problems and objective
%The relation between the previous research or the existing condition and the reference, The relation between problem definition and research variable.
%The objective answerS the problem AND solves the problems, the hypothesis is built based on the objecctives and the problems. The objective is specific.

% =========================================================
%\section{Tujuan penelitian}
% =========================================================

%Berisi tujuan penelitian yang ditulis dalam bentuk satu kalimat dan dirinci langkah-langkah nya dalam bentuk point tujuan dan berisi minimal ½ halaman.

%Objective and Hypotheses (Tujuan dan Hipotesa)
%This section should contain objective research direction (sharp and measurable) and the hypotheses (The explanation of method, concept which will be used to solve the problem as well as the reason why they will be used in the research; the explanation about the difference between method, concept, and theorem which will be used and the method, concept method, concept which were used previously ).

% =========================================================
%\section{Hipotesis}
% =========================================================
      

%Berisi prediksi hasil dan dasar prediksi yang digunakan berdasarkan referensi hasil pekerjaan/penelitian sebelumnya. Sertakan referensi yang disitasi untuk memprediksi hasil. Hipotesis Berisi minimal ½ halaman.

%\section{Scope of Work}

%Scope of work  concerns about the knowledge (reference), (facility, usability and user  OPTIONAL)


% =========================================================
%\section{Metodologi}
% =========================================================
      

%Berisi urutan langkah – langkah untuk menyelesaikan penelitian beserta teknik/metoda disetiap langkah. Buat dalam bentuk diagram blok dan deskripsikan setiap langkah tersebut. Berisi minimal 1 halaman.


% =========================================================
%\section{Research Method}
% =========================================================
    

%Berisi penjelasan singkat tentang metode/formula/skema/algoritma utama (1-2metoda) yang  akan digunakan/diusulkan dalam penelitian, berdasarkan referensi utama yang akan dijadikan acuan.
%This section contains reference tracing, requirement identification, design process, implementation process, experiment design and plan, analysis method
%This section should contain the following procedure
%1.	Reference tracing
%2.	Requirement identification
%3.	Design process
%4.	Implementation process 
%5.	Experiment design and plan (including data collection process)
%6.	Analysis/Evaluation method which will be used for analyzing the experiment result
%\section{Schedule}

%The schedule match the research method and feasible to be realized
%Timetable/Schedule (jadwal pelaksanaan)
%Activity	1	2	3	4	5	..	..
%Reference tracing							
%Requirement identification							
%Design process							
%Implementation process							
%Experiment design and plan							
%Analysis/Evaluation							

%Penelitian ini InsyaAllah direncanakan akan diselesaikan dalam tempo lima bulan. Rencana tersebut dijabarkan dalam tabel sebagai berikut:

%Pembuatan Usulan Penelitian

%Kajian Pustaka

%Desain algoritma kontrol

%Simulasi Robot

%Pengujian pada robot aktual

%Pembuatan laporan

%Seminar Hasil