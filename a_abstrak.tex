\chapter*{}
\vspace*{-1.0cm}
\begin{center}
\addcontentsline{toc}{chapter}{\protect\numberline{}ABSTRAK}
\normalfont\LARGE\textbf{ABSTRAK}
\end{center}

Menavigasi robot otonom di lingkungan yang tidak terkendali itu menantang 
karena membutuhkan seperangkat subsistem yang bekerja sama. 
Ini membutuhkan pembuatan peta lingkungan, penempatan robot di peta ini, 
membuat rencana perjalanan berdasarkan peta, menjalankan rencana ini dengan pengontrol dan tugas lainnya; 
Semua sekaligus. Di sisi lain, pertanyaan tentang navigasi otomatis sangat penting untuk masa depan robotika. 
Banyak aplikasi  memerlukan solusi ini, seperti pengiriman paket, pembersihan, pertanian, pengawasan, 
pencarian dan penyelamatan, konstruksi dan transportasi. Perhatikan bahwa semua aplikasi ini terjadi di lingkungan yang tidak diatur. 
Dalam tugas akhir ini, kami mencoba memecahkan beberapa sub-masalah yang terkait dengan navigasi otomatis di lingkungan yang tidak terkendali. 
Untuk perencanaan, kami menyajikan dan mengeksplorasi konsep dasar ruang yang diketahui, ruang bebas, 
dan asumsi tentang ruang yang tidak diketahui untuk mengatasi beberapa pengamatan lingkungan. 
Asumsi ini diperlukan untuk menavigasi dalam lingkungan yang dinamis dan asing. Untuk lingkungan yang dinamis, 
kami menyediakan dua metode pertama untuk memulihkan rintangan tiruan di peta dan membuat robot tetap mencari tujuannya. 
Untuk lingkungan yang tidak dikenal, kami menawarkan cara bernavigasi dengan lebih efisien. 
Kami melakukan simulasi Monte Carlo untuk mengevaluasi kinerja algoritma kami. 
Hasilnya menunjukkan dalam kondisi apa algoritme kami berkinerja lebih baik dan lebih buruk
Pada penelitian ini akan dirancang Autonomous Trash Collector Robot dengan arsitektur behavior based control. 

% menghindari rintangan, dan mencapai target.
\textbf{Kata kunci: Autonomous robot, Q-Learning, Behavior-Based Robot} 