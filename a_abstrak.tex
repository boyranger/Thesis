\chapter*{}
\vspace*{-1.0cm}
\begin{center}
\addcontentsline{toc}{chapter}{\protect\numberline{}ABSTRAK}
\normalfont\LARGE\textbf{ABSTRAK}
\end{center}
%[I]
Menavigasi robot otonom di lingkungan yang tidak terkendali itu menantang karena membutuhkan seperangkat subsistem yang bekerja secara bersama. Agar dapat beradaptasi dengan navigasi di lingkungan yang tidak dikenal, mobile robot harus memiliki kemampuan yang cerdas, seperti kognisi lingkungan, keputusan perilaku, dan pembelajaran. Robot kemudian akan menavigasi antara hambatan ini tanpa menabrak  dan mencapai titik tujuan tertentu.

%Navigasi pada mobile robot dengan algoritma Q-Learning. Robot mobile diharapkan mampu melakukan fungsi pencarian rute terpendek, pemetaan dan lokalisasi, dan dapat menghindari halangan di lingkungan. Penelitian ini memanfaatkan platform middleware Robot Operating System (ROS).

%[M]
Penelitian ini konsern pada navigasi otomatis pada mobile robot dari posisi awal menuju posisi tujuan. 
Untuk memecahkan beberapa sub-masalah yang terkait dengan navigasi otomatis di lingkungan yang tidak terkendali. 
Simulasi Monte Carlo dilakukan untuk mengevaluasi kinerja algoritma untuk menunjukkan dalam kondisi apa algoritma  berkinerja lebih baik dan lebih buruk.
Menggunakan kerangka kerja deep reinforcement learning,untuk mendapatkan pemetaan posisi untuk mengoptimalkan aksi pada robot mobile. Reinforcement learning memerlukan jumlah sampel pelatihan yang banyak, yang mana sangat sulit untuk dapat langsung diaplikasikan pada sekenario navigasi robot mobile secara nyata. Untuk memecahkan masalah  dilatih dilingkungan simulasi Gazebo platform middleware Robot Operating System (ROS), diikuti dengan penerapan pelatihan deep Q-Learning pada sekenario navigasi mobile robot dunia nyata.

%[Ra]
%Pada kedua simulasi dan dunia nyata eksperimen telah dilakukan untuk memvalidasi pendekatan yang diajukan. Hasil eksperimen navigasi otomatis robot mobile pada simulasi lingkungan Gazebo bahwa pelatihan DQN dapat memperkirakan fungsi nilai tindakan keadaan pada robot mobile dan menunjukan akurasi pemetaan didapat dari sensor untuk mengoptimalkan aksi robot mobile.

%Navigasi robot mobile pada lingkungan baru akan dilakukan pelatihan beberapa episode pada lingkungan simulasi untuk melihat apakah robot mobile dapat mencari rute baru ketika rute yang sudah ada tertutupi atau terhalang, sehingga robot mobile dapat menghindari halangan dinamis atau statis dengan baik. Oleh karena itu metode navigasi otomatis ini diharapkan dapat efektif dan mampu diadopsi pada berbagai lingkungan untuk robot mobile pada di lingkungan yang tidak diketahui.
%[D]
Diharapkan melalui simulasi dan percobaan, efektivitasnya dapat terbukti. Setelah dilatih dengan metode ini, robot dopat bergerak dengan aman mencapai target, navigasi di lingkungan yang tidak diketahui tanpa demonstrasi sebelumnya. Selain itu, evaluasi kuantitatif dan kualitatif yang mendalam dari metode ini disajikan dengan perbandingan dengan metode perencanaan jalur normal yang didukung oleh peta lingkungan global sebelumnya.
% menghindari rintangan, dan mencapai target.

\textbf{Kata kunci: Autonomous Robot Mobile, Q-Learning, Gazebo, Robot Operating System.} 