\chapter*{}
\vspace*{-1.0cm}
\begin{center}
\addcontentsline{toc}{chapter}{\protect\numberline{}ABSTRAK}
\normalfont\LARGE\textbf{ABSTRAK}
\end{center}

Pada penelitian ini akan dirancang Autonomous Trash Collector Robot dengan arsitektur behavior based control. Algoritma Neuro Fuzzy Q-Learning hanya diterapkan pada dua/tiga prilaku yaitu 
% menghindari rintangan, dan mencapai target.
2.
\textbf{Kata kunci: Autonomous robot, Q-Learning, Neuro Fuzzy, Behavior-Based Robot} 



%Back ground should contain reasons of why phenomenon is chosen as the research topic; 
%the gap of existing condition (previod works) and the future condition; 
%pg2 the problem identification; 
%possibility that the phenomenon will give new concepts as a result

%Berisi resume yang terdiri dari 4 paragraph yaitu : 			
%latar belakang masalah,	
%masalah utama,	
%metoda/research method yg diusulkan dan	
%hipotesis. 

%Berisi maksimal 1 halaman   	

%Purpose – This study concerns an on-line path planning technique for a behaviour-based wheeled mobile robot local navigation in an unknown environment with hurdles, using the feedforward back-propagation neural network sensor-actuator control technique. The purpose of this study is to find the non-collision path for the mobile robot moving towards the goal in a cluttered environment. 

%Design/methodology/approach – Neural network architecture input layers are the different hurdle distance information, which are acquired by an array of equipped sensors, and the output layer is the turning angle (motor control). In this way, the mobile robot is effectively being trained to move autonomously in the environment. 

%Findings – Computer simulation and real-time experimental results show that the proposed neural network controller can improve navigation performance in cluttered and unknown environments. 

%Originality/value – The proposed neural network controller gives better results (in terms of path length) as compared to previously developed models, which verifies the effectiveness of the proposed architecture.


